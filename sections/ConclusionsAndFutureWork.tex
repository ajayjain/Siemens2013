% !TEX root = ../paper.tex

\section{Conclusions and Future Work}		% 1-3 pages

\textbf{We effectively produce up to date, high resolution maps and models that assist with fast damage
assessment in disaster response, search and rescue, and indoor survivor search. Our system has an order of
magnitude better angular resolution than current satellite imaging, and each drone is under \$500,
compared to current UAVs ranging from tens to hundreds of thousands of dollars.}

\subsection{How we met our Project Goals}

After developing 3 main hardware prototypes, our final drone uses a self contained image and telemetry
collection unit, while initially we used the Arduino microprocessor.
Our system 1) captures low resolution imagery with a high altitude drone to 2) automatically identify damaged areas where 3) low altitude drones capture very high resolution imagery. Data is 4) presented on existing aerial mapping tools used by first responders.

Within dangerous buildings, quadrotors with 3D cameras capture full 3D maps of building interiors. Bodies are detected and the models are sent to doctors for remote diagnosis.
This technology saves the lives of first responders as they do not need to search inside collapsing buildings.

\subsection{Further Research}
In the future, we will investigate alternate UAV technologies to allow for more carry-weight. In turn, we could mount higher quality imaging equipment and additional sensors. After looking into other rotor based systems, we would develop an imaging platform based on a fixed-wing UAV. Fixed wing planes can cover large areas in one flight, although we would not be able to use the same system to map the interiors of buildings.

As we develop the project, we may develop an networking solution between drones without a centralized PC server. Tasks and computation would be divided between drones with peer to peer communication. This would allow for massive scale deployments of drones across a city.