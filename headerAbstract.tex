% !TEX TS-program = pdflatex
% !TEX encoding = UTF-8 Unicode
% !TEX root = ./Abstract.tex

% This is a simple template for a LaTeX document using the "article" class.
% See "book", "report", "letter" for other types of document.

\documentclass[12pt]{article} % use larger type; default would be 10pt

% Fonts
\usepackage{times}

\usepackage{setspace}
	\doublespacing

\usepackage{textcomp}

\usepackage{textcase}
\usepackage{xcolor}


\usepackage[utf8]{inputenc} % set input encoding (not needed with XeLaTeX)

%%%%%%%%%
% Page
%%%%%%%%%

\usepackage{geometry} % to change the page dimensions
	\geometry{letterpaper} % or a4paper or letterpaper (US) or a5paper or....
	\geometry{margin=1in} % for example, change the margins to 2 inches all round
	% \geometry{landscape} % set up the page for landscape

\usepackage{graphicx} % support the \includegraphics command and options
% \usepackage[parfill]{parskip} % Activate to begin paragraphs with an empty line rather than an indent

%%% PACKAGES
\usepackage{booktabs} % for much better looking tables
\usepackage{array} % for better arrays (eg matrices) in maths

\usepackage{paralist} % very flexible & customisable lists (eg. enumerate/itemize, etc.)
\usepackage{enumitem} % Remove spacing
\setlist{nolistsep}   %     between list items

\usepackage{verbatim} % adds environment for commenting out blocks of text & for better verbatim
\usepackage{subfig} % make it possible to include more than one captioned figure/table in a single float
% These packages are all incorporated in the memoir class to one degree or another...

%%% HEADERS & FOOTERS
\usepackage{fancyhdr} % This should be set AFTER setting up the page geometry
	\pagestyle{fancy} % options: empty , plain , fancy
	\renewcommand{\headrulewidth}{0pt} % customise the layout...
	\lhead{}\chead{}\rhead{}
	\lfoot{}\cfoot{\thepage}\rfoot{}


%%% SECTION TITLE APPEARANCE
%\usepackage{sectsty}
	% \allsectionsfont{\normalfont\Large\bfseries\singlespacing} % (See the fntguide.pdf for font help) \mdseries
	% (This matches ConTeXt defaults)

\usepackage{titlesec}
	\titleformat{\chapter}[display] %[display] puts the title chapter on a separate line
		{\normalfont\huge\bfseries}{\chaptertitlename\ \thechapter}{20pt}{\Huge}
	\titleformat{\section}
		{\singlespacing\normalfont\Large\bfseries}{\thesection}{1em}{\vspace{-2ex}}
	\titleformat{\subsection}
		{\singlespacing\normalfont\large\bfseries}{\thesubsection}{1em}{\vspace{-2ex}}
	\titleformat{\subsubsection}
		{\singlespacing\normalfont\normalsize\bfseries}{\thesubsubsection}{1em}{\vspace{-2ex}}

% paragraph
%\setlength{\parskip}{1cm plus4mm minus3mm}
\setlength{\parindent}{0.5cm}

\titlespacing{\paragraph}{%
  0pt}{%              left margin
  0.5\baselineskip}{% space before (vertical)
  0.5em}%               space after (horizontal)

\usepackage{titling}
	\setlength{\droptitle}{-6em}
	\posttitle{\par\end{center}\vspace{-4em}\vskip 1em\singlespacing\normalfont}



%%% ToC (table of contents) APPEARANCE
\usepackage[nottoc,notlof,notlot]{tocbibind} % Put the bibliography in the ToC
\usepackage[titles,subfigure]{tocloft} % Alter the style of the Table of Contents
	\renewcommand{\cftsecfont}{\rmfamily\mdseries\upshape}
	\renewcommand{\cftsecpagefont}{\rmfamily\mdseries\upshape} % No bold!

% Drafting
\usepackage{draftcopy}
%	\lhead{\huge  \color{black!13} {DRAFT | DRAFT | DRAFT | DRAFT} }
%	\lfoot{\huge  \color{black!13} {DRAFT | DRAFT | DRAFT | DRAFT} }

% Comments
\usepackage{comment}

% images
\DeclareGraphicsExtensions{.pdf,.png,.jpg}
\usepackage{wrapfig}
\usepackage{float}

% Bibliography
\usepackage[font={small,it}]{caption}
\usepackage[sort&compress,comma,square,super]{natbib}         	% bibliography style

%\usepackage[colorlinks]{hyperref}    	% better urls in bibliography
\usepackage{hyperref}

\renewcommand{\refname}{\vspace{-3ex}}	% no title
\bibliographystyle{IEEEtranN}			% standard formatting, ieee
%\bibliographystyle{plainnat}			% standard formatting, ieee